\documentclass{article}

\usepackage{array}
\usepackage{etoolbox}
\usepackage{fancyhdr}
\usepackage{geometry} 
\usepackage{graphicx}
\usepackage{soul}
\usepackage{titling}

%%%%%%%%%%%%%%%%%%%%%%%%%%%%%%%%%%%%%%%%%%%%%%%%%%%%%%%%%%%%
% BEGIN METADATA: Edit the following as appropriate
%%%%%%%%%%%%%%%%%%%%%%%%%%%%%%%%%%%%%%%%%%%%%%%%%%%%%%%%%%%%

\title{DeenWay}  % the title of your project
\newcommand\shorttitle{\thetitle}  % if needed: a shorter title for the document header
% Team members.
\newcommand\firstname{Rumaisa Kamran}  % full name
\newcommand\firstid{rk05969}         % ID, e.g. xy01234
\newcommand\secondname{Ammar Ahmad Rizvi} % full name
\newcommand\secondid{ar05992}        % ID, e.g. xy01234
\newcommand\thirdname{Muhammad Mehdi}  % full name
\newcommand\thirdid{mm05509}         % ID, e.g. xy01234
\newcommand\fourthname{Sakina Shabbir} % full name
\newcommand\fourthid{ss05562}        % ID, e.g. xy01234
% \newcommand\fifthname{Student 5}  % full name
% \newcommand\fifthid{id05}         % ID, e.g. xy01234

%%%%%%%%%%%%%%%%%%%%%%%%%%%%%%%%%%%%%%%%%%%%%%%%%%%%%%%%%%%%
% END METADATA: Do not edit the preamble any further.
%%%%%%%%%%%%%%%%%%%%%%%%%%%%%%%%%%%%%%%%%%%%%%%%%%%%%%%%%%%%

\pagestyle{fancy}
\lhead{Kaavish Proposal}
\chead{\shorttitle}
\rhead{Fall 2022}
\cfoot{Page \thepage}
\renewcommand{\footrulewidth}{0.4pt}

\newcommand\instruction[1]{\textit{#1}}

\begin{document}

% Cover page.
\input{cover}

%%%%%%%%%%%%%%%%%%%%%%%%%%%%%%%%%%%%%%%%%%%%%%%%%%%%%%%%%%%%
% DATA: Populate the rest of the document as instructed.
%%%%%%%%%%%%%%%%%%%%%%%%%%%%%%%%%%%%%%%%%%%%%%%%%%%%%%%%%%%%

\section{Problem definition}
% \instruction{Describe the problem that the project addresses.}
Every human is born with a state of purity and innocence, but as we proceed in our life and grow our behaviour, thoughts, attitude, speech and etc is affected by external factors. Modesty is a part of our faith which is a vital principal of Islam. Modesty manifests in many ways such as our acts and characteristics. Nowadays, major digital emerging industries are on the aim to desacralize Muslims in every way. One of the main industries that comes to our mind is social media. Through social media we are now inclined to prioritize everything which is prohibited in religion such as immodest fashion to get the engagement more in terms of followers, comments and likes, sharing of unauthentic hadiths and Qur’anic Verses without doing any search on our own, wasting our time by sharing memes and inappropriate content and etc. These social media platforms have now become a threat to community which is affecting faith in every way we can think of, a vacuum is generated where now every Muslim is feeling the need of digital platform which is totally inclined towards Muslim community and purely based on Islamic values.  
We are not scholars, but as a community it is our responsibility to provide a way or a platform for Muslims to be united and to get rid of these negative factors on different applications that are slowly destroying our faith. This application can be an effective tool for providing a platform which is free from those negative factors, creating a modest culture for Muslims, providing a platform for Da’wah, connecting scholars with everyone, maintaining contact with family and friends, giving sincere advice, inviting to religion, spreading the word of Allah and disseminating the teachings of His Prophet (SAW).

The objective of this project is to fulfil the need of every Muslim which is filtered social media application. We are now living in an age where social media has become a part of our lives and we can’t spend a single hour without seeing over feeds. Many social media giants are jumping into the race of social media but from what we have researched none of them are inclined towards Muslim Community. We are trying to fill this gap by developing a platform completely dedicated to the Muslim community and purely based on religious values. It’s a social media platform for Muslims where we have focused on two main things as for the scope of our project which includes textual presentation of posts and visual presentation of posts. As for the textual posts we will be removing islamophobia which for what we have seen never been implemented in any application, then we will be focusing on the authentication of hadiths and Qur’anic verses to restrict the user to only add those hadiths and Qur’anic verses which are authentic with references. As for Visual posts we will implementing NSFW APIS for detecting content, videos or website pages the viewer may not wish to be seen looking at in a public, formal or controlled environment. As of our secondary goal we will be implementing Friend System, Follower System, Groups and Pages will be there for users to explore, a complete search engine for Qur’anic and Hadiths, Comment and Like system for posts and Identity verification process to make the users verified for creating Pages and Groups.
\section{Social relevance}
% \instruction{Describe any societal problem that the project addresses.}
Internet is a public domain where everyone can connect and share whatever they want. Whilst social media has some tremendous positive effect on everyone whereas problems are also being identified. The online world is frequently associated with decadence, consumerism, and loose morals. These problems are connected with online harassment, abuse, toxicity, Islamophobia, sharing of nudity, promoting non-religious culture, sectarianism and fitnah. Nowadays these social media platforms have become a tool of promoting both hate and an anti-Islamic culture. One can feel that this community is deviating from the religion and losing the significance which was spread by the early generations. 
Faith is a part of our Iman, but as the modern age is emerging and being linked with our lives, we are becoming a victim of these social media platforms where a user does not even know when something immodest and immoral is going to appear to disturb its faith. There is an increasing number of unfiltered images and videos present on these platforms. It's still a policy to remove all these contents in well-known social media sites, but in the name of freedom of expression, people are still spreading unfiltered images and videos. 
When one of the original user is disturbed by a fake user, it’s a threat to privacy. Using different gender content and trying to socialize is not any type of socialization whereas it’s a crime to invade in someone personal space with false content. 
Using social media, false information about Islam is being spread both intentionally and unintentionally. Major Co-operations and individuals who poses hate towards Islam are spreading false rumours about Islam to influence the public’s point of view about religion. It can also be seen that even Muslims are sharing hadiths and qur’anic verse which are not authentic.
\section{Originality/Novelty}
% \instruction{Describe the value of solving the problem. Compare and contrast with any existing solutions.}
There are some platform which serve a similar purpose as our platform but they all have some limitations and after reviewing these platforms and identifying there problems we are trying to develop a platform which will eradicate all the issues and limitations. Similar platforms and there limitations are mentioned below:
 
 
{\textbf{Labayk - An Islamic Social Media}} \\
They claim to be the world’s first social media with a lot of features and benefits for the Muslim community. But they have failed to implement the detection and removal of Islamophobia. As far as the Media such as Photos and Videos are concerned users can upload any type of image on their profile pictures and as the security and identity of users is concerned anyone can create any page or group (Which can become a source of false information), users can easily get a verified badge based on just a small application and no system for checking the authenticity of Hadith and Qur’anic Verses.
 
{\textbf{Muslims Community}} \\
They also claim to be the world’s first Muslim community app with a lots of features and benefits for this community. This is a mobile based application so it’s only available on app store and playstore but they have failed to implement the detection and removal of Islamophobia and as far as the media such as photos and videos are concerned users can upload any type of image on their profile pictures and in their feeds there is no identification of users. No system for checking the authentication of Hadiths and Qur’anic Verses and no authentic scholars are there, one user is helping other user based on his individual preferences.
 
{\textbf{Salahuna – Islamic Social Media}} \\
This is another mobile application, where they have also claimed to be an Islamic Social Media with a lots of features but they have failed to implement the detection and removal of Islamophobia. As far as the Media such as Photos and Videos are concerned users can upload any type of images and post and as for the as the security and identity of users is concerned anyone can create any Page or group and it has no system for checking the authenticity of Hadiths and Qur’anic Verses.
 
{\textbf{Deenify – Muslim Community App}} \\
It is also an Islamic social media developed by University Student, it provides complete features which a Social Media Requires from the creation of posts to the follower and recommender system but same as other platforms this also does not provide any filtering in case of Islamophobia. In case media such as images and videos user can post anything ranging from an authentic information to any inappropriate content. There is no filtering, or restrictions imposed at the backend and there is no identity approval system.

\section{CS contribution}
% \instruction{Describe the CS component of the project, e.g. the higher level CS courses that contribute to it.}
{\textbf{Web Development:}} \\
As we are creating a Web-Based social media platform, the learnings we got from this course are very crucial for this project. \\
\\
{\textbf{Machine Learning and Artificial Intelligence:}} \\
ML and AI will help us achieve the following: \\
• Users will not be allowed to share any Qur’anic Verse or Hadith in their news feed other than groups.  (This will be done by a Similarity model based on all the verses of Quran and Six Authentic Books of Hadiths). \\
• Removal of violent, graphic content and nudity. \\
• Blurring of any adult male and female picture. In case of profile picture (for identification) user will be allowed to post only a picture with only One face. \\
• Removal of toxic factors in both comments and posts. There will be a spam filter and a recommender system. \\
• A complete implementation of Islamophobia Detection and Removal.

\section{Scope and Deliverables}
% \instruction{Justify the scope of the project with respect to the size of the team and the year long duration. List the foreseeable deliverables.}
This project is based on a social media web application in which we will try to eliminate these immodest, immoral and irritating factors by doing spam filtering, restriction on nudity, removal of unwanted content. Users will be able create their profiles, connect with others, but they would be allowed to post any toxic/hate comment. Authentic groups will be created after a short process of approval in which users will be able to ask for the guidance. These groups will be monitored by admins. Users will not be allowed to post any hadiths and Quranic ayahs, only group Admins will be allowed to share hadiths and Quranic Ayahs.
We will have a complete web application with all CRUD operations. With a sophisticated frontend and backend with REST API’s. We will make our app intelligent by all the machine learning algorithms. People will be able to enjoy this application with all the content free of any negativity, hate speech, nudity, immodesty etc. We will be able to create an online community built on moral principles. This application will be family friendly without any kind of toxicity. A platform will be established with exclusion of any other kind of immoral and immodest content.

\section{Feasibility}
% \instruction{List the resources, e.g. datasets, compute resources, software libraries, hardware, required for the project. Mention how you expect to access and utilize them for the project.}
We will be using an extensive list of resources for this project which are as follows: \\
\\
{\textbf{Data Sets:}}  \\
 Since the scope of our project heavily relies on the use of datasets, we have chosen to pick datasets right off popular social media platforms like Twitter and Reddit. These platforms have been vastly associated with widespread Islamophobia from users and many groups that have been massively engaged in conducting hate speech against Islam, proclaiming it to be an oppressive, monolithic, sexist, violent and aggressive religion. To detect weak and strong groups of Islamophobia, we aim to include a number of tweets from even before 2020 which will be acquired through the Twitter Rest Api. \\
 \\
{\textbf{Other Resources and Software Libraries:}}  \\
• HTML5: Front-end development. \\
• CSS3: Design and structural development of HTML elements. \\
• React JS: Front-end JavaScript Framework. \\
• Bootstrap: Front-end development. \\
• Axios: JavaScript Library (popular amongst platforms like React JS). \\
• MomentJS: Works well with date-time objects in displaying and manipulating them. A JavaScript framework. \\
• Django Rest Framework: Python based web framework. (Back-end development). \\
 \\
{\textbf{Access and Utilisation:}} \\
Most of the resources we will be using are free, easily available and open source. We will be using our personal devices to work on the project.
\section{Team dynamics}
% \instruction{Justify the suitability of the team members to the project. For example, their relevant courses, projects, internships, or research.}
• {\textbf{Ammar}} - He has a good command on front end side of the project which includes html, css and other features. Secondly, he is currently doing some online courses related to Artificial Intelligence and Machine Learning. \\
\\
• {\textbf{Rumaisa}} – She will be working on back end side of the project which includes database connectivity with different data sets. Also, she is currently taking online courses on Artificial Intelligence and Machine Learning. \\
\\
• {\textbf{Mehdi}} - He is currently taking an online course on Artificial Intelligence since it is integral to our project. He has a strong command on database systems and hence is looking forward to work on that part of the project. \\
\\
• {\textbf{Sakina}} - She has a strong command on research projects and since our project involves searching for relevant data sets to work on, she thinks this project will be a good fit for her. In addition to this, she plans on taking Machine Learning as an online course on Coursera this semester to further develop an understanding of how datasets can be used in project development. \\
\section{References}
% \instruction{List your references.}
{\textbf{1.}}	Blaker, Lisa (2016) "The Islamic State’s Use of Online Social Media," Military Cyber Affairs: Vol. 1 : Iss. 1 , Article 4. http://dx.doi.org/10.5038/2378-0789.1.1.1004 Available at: https://digitalcommons.usf.edu/mca/vol1/iss1/4
\\
{\textbf{2.}}	Slama, M. (2017), A subtle economy of time: Social media and the transformation of Indonesia's Islamic preacher economy. Economic Anthropology, 4: 94-106. https://doi.org/10.1002/sea2.12075 \\
{\textbf{3.}}	M. T. Islam , "THE IMPACT OF SOCIAL MEDIA ON MUSLIM SOCIETY: FROM ISLAMIC PERSPECTIVE", International Journal of Social And Humanities Sciences, vol. 3, no. 3, pp. 95-114, Dec. 2019. \\
{\textbf{4.}}	Qayyum, A., & Mahmood, Z. (2015). Role of social media in the light of Islamic teaching. Al-Qalam, 20(2), 26-35. \\
{\textbf{5.}}	Mowlana, H. (2003). Foundation of communication in Islamic societies. Mediating religion: Conversations in media, religion and culture, 305-316. \\
{\textbf{6.}}	Halim, W. (2018). Young Islamic preachers on Facebook: Pesantren As’ adiyah and its engagement with social media. Indonesia and the Malay World, 46(134), 44-60. \\
{\textbf{7.}}	Laluddin, H. (2014). Conception of society and its characteristics from an Islamic perspective. International Journal of Islamic Thought, 6, 12. \\
{\textbf{8.}}	Akhavan, N. (2009). Social Media and the Islamic Republic. Social media in Iran: Politics and society after, 213-230. \\
{\textbf{9.}}	Ibahrine, M. (2020). Islam and social media.\\
{\textbf{10.}}	Al Kahlout, N. I. A. F. (2012). social media & its effects on decision making of senior management: case study islamic university of gaza.\\
{\textbf{11.}}	Fakhruroji, M. (2019). Digitalizing Islamic lectures: Islamic apps and religious engagement in contemporary Indonesia. Contemporary Islam, 13(2), 201-215.\\
{\textbf{12.}}	Shabbir, M. S., Ghazi, M. S., & Mehmood, A. R. (2016). Impact of social media applications on small business entrepreneurs. Arabian Journal of Business and Management Review, 6(3), 203-05.

% External advisor undertaking.
\input{external}

\end{document}

%%% Local Variables:
%%% mode: latex
%%% TeX-master: t
%%% End:
